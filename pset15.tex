\documentclass[a4paper]{exam}

\usepackage{amsfonts,amsmath,amsthm}
\usepackage{geometry}


\title{Problem Set 15: Graphs and Matchings}
\author{CS/MATH 113 Discrete Mathematics}
\date{Spring 2024}

\boxedpoints

\printanswers

\begin{document}
\maketitle

The problems below make use of concepts and definitions from Sections 10.1 and 10.2 in our textbook. Solving the problems will be tremendously easier if you have gone over the sections and the worked examples included therein, browsed the end-of-section exercises, and consulted their solutions at the back of the book. If you are still stuck at some problem, feel free to consult course staff during their consultation hours as shared on Canvas.

\begin{questions}
  
  \question Prove that a connected graph with $n$ vertices has at least 2 vertices of the same degree.
  \begin{solution}
    % Enter your solution here.
    \renewcommand\qedsymbol{$\square$}
    \begin{proof}
      We attempt a proof by contradiction. \\
      Assume that a every vertex in a connected graph has a distinct degree. This means that the degrees of the vertices are $0, 1, 2, \ldots, n-1$ for n vertices. \\
      This assumption implies that the node with degree zero will be isolated and not connected to any other node. \\
      And since the graph is connected, the node with degree $n-1$ must be connected to all other nodes implying that each node in a connected graph has a degree of atleast one. \\
      This creates a contradiction and therefore proves that a connected must have atleast 2 vertices of the same degree.
    \end{proof}
  \end{solution}

  \question Prove that any cycle graph with an even number of vertices can be 2-colored.
  \begin{solution}
    % Enter your solution here.
    \renewcommand\qedsymbol{$\square$}
    \begin{proof}
      We attempt a proof by structural induction. \\
      \textbf{Base Case:} For a cycle graph with 4 vertices, we can color the vertices alternately with two colors. So the base case holds. \\
      \textbf{Inductive Hypothesis:} Assume that a cycle graph with $2n$ vertices can be 2-colored. \\
      \textbf{Inductive Step:} Consider a cycle graph with $2(n+1)$ vertices. We can add the two additional vertices to the cycle graph in such a way that they divide the cycle into two smaller sub-cycles. \\
      Then these two vertices with the color of their respective cycles maintaining the 2-colored property of the cycle graph. \\
      This completes the proof.
    \end{proof}
  \end{solution}

  \question Prove by induction that a graph with $n$ edges requires no more than $n+1$ colors to ensure that no two adjacent edges have the same color.
  \begin{solution}
    % Enter your solution here.
    \renewcommand\qedsymbol{$\square$}
    \begin{proof}
      Attempting a proof by mathematical induction. \\
      \textbf{Base Case:} For a graph with 2 edges, we can color the edges with 2 colors. Since $2 \leq 2+1$, the base case holds. \\
      \textbf{Inductive Hypothesis:} Assume that a graph with $n$ edges can be colored with no more than $n+1$ colors. \\
      \textbf{Inductive Step:} Consider a graph with $n+1$ edges. We need only $n+1$ colors to ensure that each adjacent edge has a different color. \\
      As per our hypothesis we have $n+1+1$ colors so and and that graph has $n+1$ edges so the hypothesis holds i.e $(n+1) \leq (n+1)+1$. \\
      This completes the proof.
    \end{proof}
  \end{solution}
  
  \question A \textit{perfect matching} in a graph is a matching that covers every vertex of the graph. Prove that if all the vertices of a bipartite graph have the same degree, then it has a perfect matching.
  \begin{solution}
    % Enter your solution here.
    \renewcommand\qedsymbol{$\square$}
    \begin{proof}
      Atempting a direct proof. \\
      Consider a bipartite graph $G = (V,E)$. This graph according to the definition of a bipartite graph can be broken down into two disjoint sets $V_1$ and $V_2$. \\
      Every vertex in $V_1$ completely maps onto a vertex in $V_2$ and vice versa and that both seta have the same cardinality. \\
      This mapping effectively covers all the vertices in the graph creating a perfect matching with every vertex having the same degree.
    \end{proof}
  \end{solution}

  \question 100 tourists have arrived at Mohenjodaro. 25 tour guides are available for a one-on-one tour. Each tourist likes at least 10 of the guides. Show that the tours can be arranged such that each tourist tours with a guide that they like and no guide gives more than 10 tours.

  \underline{Hint}: Consider each guide having 10 time slots corresponding to the constraint that each guide conducts at most 10 tours.
  \begin{solution}
    % Enter your solution here.
    \renewcommand\qedsymbol{$\square$}
    \begin{proof}
      Attempting a proof using Hall's Marriage Theorem. \\
      Let $T$ be the set of tourists and $G$ be the set of tour guides. \\
      Let $N(S)$ be the set of tour guides that are liked by the tourists in $S$. \\
      We need to show that $|N(S)| \geq |S|$ for all $S \subseteq T$. \\
      Consider a subset $S$ of tourists. The number of guides that are liked by the tourists in $S$ is at least $10|S|$. \\
      Since each guide can conduct at most 10 tours, the number of tours conducted by the guides in $N(S)$ is at most $10|N(S)|$. \\
      This implies that $10|N(S)| \geq 10|S| \Rightarrow |N(S)| \geq |S|$. \\
      This satisfies the condition of Hall's Marriage Theorem and implies that there exists a perfect matching between the tourist and the guides such that no guide gives more than 10 tours.
    \end{proof}
  \end{solution}

\end{questions}
\end{document}
%%% Local Variables:
%%% mode: latex
%%% TeX-master: t
%%% End: